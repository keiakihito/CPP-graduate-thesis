\chapter{Introduction}\label{ch:introduction}

Modern music streaming platforms rely heavily on machine learning to support search, recommendation, and discovery.
Much of the prior work in this area, as well as many industrial systems, is built around large-scale collaborative filtering and deep models trained on extensive user interaction logs.
While effective for mainstream catalogs, these approaches are ill-suited to small, curated archives that lack dense user data and whose mission emphasizes exploration over pure popularity.
Classical music archives such as the iPalpiti collection exemplify this setting: they contain high-quality, stylistically coherent recordings but limited usage data, and they require tools that can surface musically meaningful relationships among works, performances, and artists.

In such environments, content-based recommendation driven by audio embeddings becomes central.
Recent advances in pretrained audio models---including lightweight convolutional networks and large transformer-based architectures---enable the extraction of rich musical representations from raw audio.
However, most evaluations of these models are conducted on large-scale, heterogeneous datasets, where increased model capacity often correlates with improved benchmark performance.
It remains unclear whether this trend transfers to small, single-domain archives such as classical music collections.

In small-scale settings, limited data diversity, constrained candidate sets, and coarse proxy definitions may reduce the practical benefit of increased model complexity.
This raises a fundamental design question for content-based recommendation systems in archival contexts: does increasing embedding-model capacity meaningfully improve similarity retrieval performance, or does it introduce diminishing returns under constrained conditions?


\section{Approach and Scope}

This thesis addresses this question by focusing exclusively on the backend: the stage that maps audio recordings to embeddings and supports similarity search, independent of end-to-end user modeling.

The study adopts a hypothesis-driven experimental design inspired by research-oriented test-driven development (TDD).
Rather than assuming that larger models necessarily yield better retrieval performance, we begin with the hypothesis that increased embedding-model capacity may not improve similarity retrieval in small-scale, single-domain archives.

To evaluate this hypothesis, we systematically assess retrieval performance across a spectrum of model capacities, treating architecture size as the primary independent variable.
Proxy retrieval tasks are formulated to reflect musically meaningful similarity, including both sanity-check retrieval tests and a musical-character-based proxy (e.g., \emph{heavy} vs.\ \emph{light}).

Ranking quality is measured using information-retrieval metrics such as NDCG, Precision@K, and Recall@K.
Experimental conditions are systematically varied to identify scenarios in which higher-capacity models either fail to provide measurable gains or demonstrate meaningful improvements over lighter architectures.

\section{Research Questions}

RQ1: In a small-scale classical music archive (iPalpiti), how does embedding-model capacity influence content-based music similarity retrieval performance?

RQ2: Under what conditions, if any, do higher-capacity embedding models provide meaningful gains over lighter architectures?


\paragraph{Hypothesis.}
In a small-scale, single-domain classical music dataset, increasing embedding-model capacity does not necessarily improve music-similarity retrieval performance.


\section{Contributions}

This work makes three main contributions.

% Before final results:
First, it provides a systematic empirical investigation of the relationship between embedding-model capacity and content-based similarity retrieval performance in a small-scale, single-domain classical archive.

% After confirming results:
% First, it provides empirical evidence challenging the assumption that increased embedding-model capacity correlates with improved content-based similarity retrieval in small-scale, single-domain classical archives.

Second, it establishes musically grounded evaluation criteria (via structured proxy retrieval tasks) that enable the identification of conditions under which retrieval improvements are meaningful.

Third, through a hypothesis-driven experimental design, it analyzes how embedding-model capacity behaves under constrained archival conditions, clarifying the practical limits of model complexity in small, single-domain recommendation settings.


\section{Thesis Organization}

The remainder of the thesis is organized as follows.
Chapter~2 reviews related work on audio embeddings, evaluation strategies, and system design for music recommendation.
Chapter~3 formalizes the problem setting and describes the overall backend pipeline.
Chapter~4 details the methodology and experimental design.
Chapter~5 presents and analyzes the results.
Chapter~6 discusses implications, limitations, and directions for future work.
