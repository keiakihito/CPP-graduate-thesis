\chapter{Discussion and Conclusion}

\section{Overview of Findings}
% Briefly restate the purpose of the study and summarize the main findings
% from Chapter 5 at a high level, without repeating numerical results.

\section{Implications for Backend Design in Classical Music Archives}
% Discuss what the observed results imply for designing
% embedding-based backends in classical music archives such as iPalpiti.
% Focus on model family selection and practical trade-offs.

\section{Limitations}
All conclusions are conditioned on Condition Z (Chapter~3): a small-scale, single-domain classical music archive and offline, proxy-based evaluation.
The findings of this thesis are subject to several limitations that are intrinsic to its scope and design.
All conclusions are conditioned on a small-scale, single-domain classical music archive and on the use of offline, proxy-based evaluation rather than real user interaction data.
As such, the results should not be interpreted as general claims about the effectiveness of embedding model families across large, heterogeneous music catalogs or fully personalized recommendation systems.


\section{Directions for Future Work}

One promising direction for future work is the exploration of discovery-oriented similarity beyond the proxy tasks considered in this thesis.
In particular, segment-level embeddings and more expressive aggregation strategies may enable the identification of musically meaningful relationships that are not captured by coarse proxy definitions.
Such exploratory analysis could support serendipitous discovery in classical music archives, but falls outside the scope of the present study, which focuses on isolating embedding model capacity under constrained, proxy-based evaluation.


\section{Conclusion}
% Provide a concise concluding statement.
% Revisit Research Question 1 and summarize how it was addressed.
% Emphasize the contribution and potential impact of the work.
